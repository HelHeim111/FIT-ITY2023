\documentclass[a4paper, 11pt]{article}

\usepackage[left=2cm,text={17cm,24cm},top=3cm]{geometry}
\usepackage{times}
\usepackage[czech]{babel}
\usepackage[unicode]{hyperref}


\begin{document}
\begin{titlepage}
    \begin{center}
        {\Huge\textsc{Vysoké učení technické v Brně} \\}
        \huge
        \textsc{Fakulta informačních technologií}\\
        \vspace{\stretch{0.382}}
            \LARGE
            Typografie a publikování -- 4. projekt \\
            \Huge
            Bibliografie
        \vspace{\stretch{0.618}}
    \end{center}
    {\Large 31. března 2023 \hfill Denys Petrovskyi}
\end{titlepage}

\section{Úvod}
Počátek typografie je obvykle spojován se vznikem knihtisku, který kolem roku 1440 vynalezl Johannes Gutenberg.
Typografie označuje dovednost a umělecký přístup k uspořádání psaného jazyka způsobem, 
který je vizuálně jasný, snadno čitelný a atraktivní pro diváky. Tento proces zahrnuje pečlivý výběr řezů písma, 
velikostí bodů, délek řádků, sledování (řádkování) a mezer mezi jednotlivými písmeny. viz \cite{OdbornyText}
Více o historii typografie se můžete dozvědět zde \cite{BakalarPrace1}
Fakta o typografii viz \cite{Article}
\section{\LaTeX}

\LaTeX je systém pro přípravu dokumentů, který je založen na sázecím systému 
\TeX vyvinutém Donaldem Knuthem na konci 70. let a značkovacím jazyce, který je 
široce používán pro vytváření vědeckých a technických dokumentů. \LaTeX umožňuje 
autorům soustředit se na obsah jejich dokumentu, přičemž se stará o formátování, 
rozvržení a další typografické aspekty. \LaTeX je populární na akademické půdě a ve 
vědecké komunitě, kde se používá k vytváření dokumentů, jako jsou výzkumné práce, 
diplomové práce, zprávy a knihy. viz \cite{WhyLatex}. \LaTeX může vytvořit různé krásné vzory 
a rozvržení pro vaše životopisy nebo články. \cite{FullArticle}

\section{Příkazy v \LaTeX}
Příkazy jsou základní funkcí LaTeXu, která vám umožňuje formátovat a strukturovat vaše dokumenty. 
Jsou to klíčová slova, která říkají LaTeXu, aby provedl konkrétní akci nebo změnil vzhled vašeho textu. 
Zde jsou některé běžné typy příkazů v LaTeXu:
    \begin{itemize}
        \item Formátovací příkazy: Tyto příkazy se používají ke změně vzhledu vašeho textu. 
        Například \verb|\textbf{}| udělá váš text tučným, zatímco \verb|\underline{}| váš text podtrhne. 
        Mezi další formátovací příkazy patří \verb|\textit{}| pro kurzívu, \verb|\texttt{}| pro písmo psacího stroje a \verb|\sout{}| pro 
        přeškrtnutý text. viz \cite{Overleaf}
        \item Příkazy dělení: Tyto příkazy se používají 
        ke strukturování dokumentu do sekcí, kapitol a podsekcí. Mezi nejčastěji 
        používané příkazy pro dělení sekcí patří \verb|\section{}|, 
        \verb|\subsection{}| a \verb|\chapter{}|.viz \cite{Stack}
        \item Příkazy seznamu: LaTeX poskytuje několik příkazů pro vytváření seznamů. 
        Prostředí \verb|itemize| vytváří seznamy s odrážkami, zatímco prostředí \verb|enumerate| vytváří číslované seznamy. 
        Pomocí prostředí popisu můžete také vytvářet vlastní seznamy. viz \cite{Latexlist}
        \item Matematické příkazy: LaTeX je široce používán v matematice a vědě, 
        protože poskytuje silnou podporu pro sazbu matematických rovnic. Příkazy pro 
        matematické symboly a operátory zahrnují \verb|\frac{}{}|, \verb|\sqrt[]{}| a \verb|\sum_{i=1}^{n} {}|.
        Také v latexu jsou příkazy pro teorii pravděpodobnosti. Vice viz \cite{Bakalarska2}

    \end{itemize}
\newpage

\bibliographystyle{czechiso}
\renewcommand{\refname}{Literatura}
\bibliography{proj4}

\end{document}