\documentclass[a4paper, twocolumn, 11pt]{article}

\usepackage[utf8]{inputenc}
\usepackage{times}
\usepackage[left=1.4cm, text={18.2cm,25.2cm}, top=2.3cm]{geometry}
\usepackage[IL2]{fontenc}
\usepackage[czech]{babel}
\usepackage[utf8]{inputenc}
\usepackage[hidelinks]{hyperref}
\usepackage{amsthm}
\usepackage{amssymb}
\usepackage{amsmath}

\theoremstyle{definition}
\newtheorem{definice}{Definice}

\theoremstyle{definition}
\newtheorem{veta}{Věta}

\begin{document}
    \begin{titlepage}
        \begin{center}
            {\Huge\textsc{Vysoké učení technické v Brně} \\ [0.5em]}
            \huge
            \textsc{Fakulta informačních technologií}\\
            \vspace{\stretch{0.382}}
                \LARGE
                Typografie a publikování -- 2. projekt \\ [0.4em]
                Sazba dokumentů a matematických výrazů
            \vspace{\stretch{0.618}}
        \end{center}
        {\Large 2023 \hfill Denys Petrovskyi (xpetro27)}
    \end{titlepage}

    \section*{Úvod}\label{page}

    V této úloze si vyzkoušíme sazbu titulní strany, matematických vzorců, 
    prostředí a dalších textových struktur obvyklých pro technicky zaměřené 
    texty -- například Definice \ref{def} nebo rovnice \eqref{rov_3} na straně \pageref{page}.
    Pro vytvoření těchto odkazů používáme kombinace příkazů \verb|\label|, \verb|\ref|, \verb|\eqref| a \verb|\pageref|.
    Před odkazy patří nezlomitelná mezera. Pro zvýrazňování textu jsou zde několikrát použity příkazy \verb|\verb| a \verb|\emph|.
    \par Na titulní straně je použito prostředí \verb|titlepage| a sázení nadpisu podle optického středu s využitím přesného zlatého řezu.
    Tento postup byl probírán na přednášce. Dále jsou na titulní straně použity čtyři různé velikosti písma a mezi dvojicemi 
    řádků textu je použito odřádkování se zadanou relativní velikostí 0,5 em a 0,4 em\footnote[1]{Nezapomeňte použít správný typ mezery mezi číslem a jednotkou.}.

    \section{Matematický text}

    V této sekci se podíváme na sázení matematických symbolů a výrazů v plynulém textu pomocí prostředí \verb|math|.
    Definice a věty sázíme pomocí příkazu \verb|\newtheorem| s využitím balíku \verb|amsthm|.
    Někdy je vhodné použít konstrukci \verb|${}$| nebo \verb|\mbox{}|, která říká, že (matematický) text nemá být zalomen.

    \begin{definice} \label{def}
        Zásobníkový automat \emph{(ZA) je definován jako sedmice tvaru, $A = (Q, \Sigma, \Gamma, \delta, q_0, Z_0, F)$ kde:}
        \begin{itemize}
            \item \emph{$Q$ je konečná} množina vnitřních (řídicích) stavů,
            \item \emph{$\Sigma$ je konečná} vstupní abeceda,
            \item \emph{$\Gamma$ je konečná} zásobníková abeceda,
            \item \emph{$\delta$ je} přechodová funkce $Q\times(\Sigma\cup\{\epsilon\})\times\Gamma\rightarrow2^{Q\times\Gamma^{\ast}}$,
            \item $q_0 \in Q$ \emph{je} počáteční stav, $Z_0\in\Gamma$ \emph{je} startovací symbol zásobníku a $F \subseteq Q$ \emph{je} množina koncových stavů.
        \end{itemize}
        \par Nechť $P = (Q, \Sigma, \Gamma, \delta, q_0, Z_0, F)$ je ZA. \emph{Konfigurací} nazveme trojici $(q, w, \alpha) \in Q\times\Sigma^{\ast}\times\Gamma^{\ast}$, 
        kde $q$ je aktuální stav vnitřního řízení, $w$ je dosud nezpracovaná část vstupního řetězce a $\alpha = Z_{i_1} Z_{i_2} \dots Z_{i_k}$ je obsah zásobníku.
    \end{definice}

    \subsection{Podsekce obsahující definici a větu}
    
    \begin{definice}
        Řetězec $w$ nad abecedou $\Sigma$ je přijat ZA \emph{A~jestliže} $(q_0, w, Z_0) \underset{A}{\overset{\ast}{\vdash}} (q_F, \epsilon, \gamma)$ 
        \emph{pro nějaké} $\gamma\in\Gamma^{\ast}$ a $q_F \in F$.
        \emph{Množina} $L(A) = \{w \mid w$ \emph{je přijat ZA} $A\} \subseteq\Sigma^{\ast}$ \emph{je} jazyk přijímaný ZA $A$.
    \end{definice}

    \begin{veta}
        \emph{Třída jazyků, které jsou přijímány ZA, odpovídá} bezkontextovým jazykům.
    \end{veta}

    \section{Rovnice}
    
    Složitější matematické formulace sázíme mimo plynulý text pomocí prostředí \verb|displaymath|.
    Lze umístit i několik výrazů na jeden řádek, ale pak je třeba tyto vhodně oddělit, například příkazem \verb|\quad|.
    $$1^{2^3} \neq \Delta_{\Delta_{\Delta^3}^2}^1 \quad y_{22}^{11} - \sqrt[9]{x + \sqrt[7]{y}} \quad x > y_1 \leq y^2$$
    V rovnici \eqref{rov_2} jsou využity tři typy závorek s různou \emph{explicitně} definovanou velikostí. 
    Také nepřehlédněte, že nasledující tři rovnice mají zarovnaná rovnítka, a použijte k tomuto účelu vhodné prostředí.
    \begin{eqnarray}
        -\cos{\beta}^2 = \frac{\frac{\frac{1}{x} + \frac{1}{3}}{y} + 1000}{\prod\limits _{j=2}^8 q_j}\\
        \bigg(\Big\{b \star \big[3 \div 4 \big] \circ a \Big\}^{\frac{2}{3}} \bigg) = \log_{10} x\label{rov_2}\\
        \int_a^b f(x)\, \mathrm{d}x = \int_c^d f(y)\, \mathrm{d}y \label{rov_3}
    \end{eqnarray}

    V této větě vidíme, jak vypadá implicitní vysázení limity $\lim_{m\to\infty} f(m)$ v normálním odstavci textu.
    Podobně je to i s dalšími symboly jako $\bigcup_{N\in\mathcal{M}} N$ či $\sum_{i=1}^m x_{i}^2$.
    S vynucením méně úsporné sazby příkazem \verb|\limits| budou vzorce vysázeny v podobě 
    $\lim\limits _{m\to\infty} f(m)$ a $\sum\limits _{i=1}^m x_{i}^4$.

    \section{Matice}

    Pro sázení matic se velmi často používá prostředí \verb|array| a závorky (\verb|\left|, \verb|\right|).
    $$
        \mathbf{B} = 
        \left|
            \begin{array}{cccc}
                b_{11} & b_{12} & \cdots & b_{1n}\\
                b_{21} & b_{22} & \cdots & b_{2n}\\
                \vdots & \vdots & \ddots & \vdots\\
                b_{m1} & b_{m2} & \cdots & b_{mn}\\
            \end{array}
        \right| =
        \left|
            \begin{array}{cc}
                t & u\\
                v & w\\
            \end{array}
        \right| = tw - uw
    $$

    $$
        \mathbb{X} = \mathbf{Y} \Longleftrightarrow 
        \left[
            \begin{array}{ccc}
                & \Omega + \Delta & \psi\\
                \overrightarrow{\pi} & \omega & \\  
            \end{array}
        \right] \neq 42
    $$

    Prostředí \verb|array| lze úspěšně využít i jinde, například na pravé straně následující rovnice. 
    Kombinační číslo na levé straně vysázejte pomocí příkazu \verb|\binom|.

    $$
        \begin{pmatrix}
            n\\
            k
        \end{pmatrix}
        = 
        \left\{
            \begin{array}{c l}
                0 & \text{pro } k < 0\\
                \frac{n!}{k!(n-k)!} & \text{pro } 0 \leq k \leq n\\
                0 & \text{pro } k > 0\\
            \end{array}
        \right.
    $$

\end{document}